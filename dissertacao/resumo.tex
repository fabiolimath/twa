%%%%%%%%%%%%%%%%%%%%%%%%%%%%%%%%%%%%%
%% Resumo
%% Copyright 2009 Fabio de Oliveira Lima.
%% Este documento � distribu�do nos termos da licen�a 
%% GNU General Public License v2.
%%%%%%%%%%%%%%%%%%%%%%%%%%%%%%%%%%%%%

\begin{resumo}

% Apresenta��o concisa dos pontos relevantes, dando uma visao rapida e
% clara do conte�do do trabalho.


 Este trabalho apresenta um novo modelo de programa��o linear inteira-mista para o projeto de redes �pticas de comunica��o. Trata-se de uma 
modelagem ampla, que engloba o projeto da topologia l�gica da rede, o roteamento das demandas de tr�fego, al�m do roteamento e aloca��o de 
comprimento de onda aos caminhos �pticos. A formula��o suporta m�ltiplas liga��es entre cada par de n�s da rede, seja na topologia f�sica ou 
virtual. Em sua vers�o b�sica, o modelo minimiza os custos de instala��o da rede f�sica e o custo de opera��o da rede projetada. No entanto, sua formula��o
permite a que sejam exploradas diversas m�tricas, como o congestionamento da rede, que foi utilizado para compara��o com resultados da literatura.
Neste trabalho s�o apresentados resultados de experimentos com o objetivo de validar a efici�ncia desta formula��o com rela��o � qualidade das solu��es e
desempenho computacional de trabalhos anteriores sobre o mesmo assunto. Tamb�m � apresentada uma nova forma, muito eficiente, de se obter \textit{lower bounds}
para o congestionamento.

\end{resumo}
