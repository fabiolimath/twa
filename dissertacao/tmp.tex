


Por simplicidade, � assumido que todos as liga��es l�gicas possuem a mesma capacidade de tr�fego.

Todos os n�s da rede s�o equipados com OXCs (\textit{Optical Cross-Connect}) \cite{Zang00}.

N�o � considerada capacidade de convers�o de comprimentos de onda, dessa forma, em todas as fibras por onde uma liga��o l�gica passa, ela deve utilizar o
mesmo comprimento de onda.

N�o ser�o consideradas aqui a
possibilidade de bloqueio de pacotes e nem outros tipos de perdas na transmiss�o. Portanto, � assumido que todo o tr�fego da rede ser� devidamente enviado e
recebido.


O VTD consiste na determina��o de quais n�s ser�o interligados diretamente. A topologia l�gica � base para a solu��o do problema de distribui��o de
tr�fego. Uma
solu��o deste problema consiste em determinar uma topologia l�gica e a forma como as demandas de tr�fego ser�o escoadas atrav�s da concatena��o das
liga��es l�gicas. 
