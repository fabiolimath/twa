
% \pretextualchapter{}
\chapter*{Ferramentas Computacionais}
% \newpage


As ferramentas computacionais envolvidas neste trabalho, listadas abaixo, s�o distribu�das sob licensas de C�digo Livre (\textit{Open Source}). O c�digo
fonte $\LaTeX$ desta disserta��o e todo o trabalho desenvolvido est� dispon�vel em \url{http://code.google.com/p/twa}. 

Todas as figuras incluidas neste texto foram criadas em SVG
(\textit{Scalable Vectorial Graphics} - \url{http://w3.org/Graphics/SVG}) e convertidas para o formato EPS (\textit{Encapsulated PostScript} -
\url{http://adobe.com/products/postscript}) para posterior inclus�o no c�digo $\LaTeX$, ambos formatos abertos. A Figura \ref{fig:nsfnet}, criada pelo autor
deste
texto, est� registrada em \url{http://wikimedia.org/wiki/File:NSFNET_14nodes.svg}.

\begin{itemize}
   \item Kubuntu GNU/Linux: a vers�o 9.10 foi usada na esta��o de trabalho e a vers�o 9.04 no servidor aonde foram
executados os testes computacionais. 
\url{http://kubuntu.org}

   \item GLPK 4.37 - \textit{GNU Linear Programming Kit}: usado para resolver modelos de programa��o inteira e converter c�digo AMPL em FreeMPS.
\url{http://gnu.org/software/glpk}

   \item SCIP - \textit{Solving Constraint Integer Programs}, vers�o 1.1.0 Linux X86: usado para resolver os modelos de programa��o interira mista.
\url{http://scip.zib.de}

   \item CLP 1.11 - \textit{Coin-or Linear Programming}: usado internamente pelo SCIP para resolver subproblemas de programa��o linear.
\url{http://coin-or.org}

   \item TexLive 2007: distribui��o $\LaTeX$ utilizada para a confec��o desta disserta��o.\\
\url{http://tug.org/texlive}

   \item Kile 2.0.83: editor de texto com ferramentas para autoria em $\LaTeX$ utilizado. \\
\url{http://kile.sourceforge.net}

   \item Inkscape 0.47: editor de desenho vetorial utilizado para criar as figuras SVG e convert�-las em EPS. 
\url{http://inkscape.org}

\end{itemize}
 

% \vspace{2cm}
% \vspace{5cm}

\begin{center}
\textbf{Feito em} \\
\textbf{$\LaTeX$}
\end{center}
